% !TeX root = ../main.tex

\chapter{Introduction}
\section{What is this Template About?}
This is a LaTeX template for NTU (National Taiwan University) master's thesis, specifically designed for the LaDS (Laboratory of Dependable Systems). It adheres to the NTU thesis formatting guidelines and incorporates features commonly used in LaDS research works.


\section{Example of References}
Here is an example of a reference citation \cite{brier2004correlation}. To add references, please edit the `back/references.bib` file.

\section{Example of Figures}
An example of a figure is shown in Figure~\ref{fig:communication-scheme-hardware}.

It is recommended to store all figures in the `figures/` folder. And with the format of PDF.

\begin{figure}[h!]
    \centering
    \includegraphics[width=0.9\textwidth]{figures/communication-scheme-hardware.pdf}
    \caption{The communication scheme and hardware setup.}
    \label{fig:communication-scheme-hardware}
\end{figure}


\section{Example of Tables}

An example of a table is shown in Table~\ref{tab:tvla-ddrla-comparison}.



\vspace{1em}
\begin{table}[h!]
    \centering
    \caption{Comparison between TVLA and DDR-LA (Proposed Method).}
    \label{tab:tvla-ddrla-comparison}
    \begin{tabular}{@{} l l l @{}}
        \toprule
        & \textbf{TVLA} & \textbf{DDR-LA (Proposed Method)} \\
        \midrule
        \makecell[l]{(Power Trace) \\Data Processing}
        & None
        & \makecell[l]{Uses a deep learning model} \\
        \addlinespace
        Statistical Test
        & \makecell[l]{Welch's t-test \\(on raw data)}
        & \makecell[l]{Welch's t-test \\(on reduced data)} \\
        \addlinespace
        Capability
        & \makecell[l]{Detects simple, univariate\\ leakage}
        & \makecell[l]{Detects higher-order \&\\ multivariate leakage} \\
        \bottomrule
    \end{tabular}
\end{table}