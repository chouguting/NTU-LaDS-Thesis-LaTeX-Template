% !TeX root = ./main.tex

% --------------------------------------------------
% 資訊設定(Information Configs)
% --------------------------------------------------

\ntusetup{
  university*   = {National Taiwan University},
  university    = {國立臺灣大學},
  college       = {電機資訊學院},
  college*      = {College of Electrical Engineering and Computer Science},
  institute     = {電機工程學系},
  institute*    = {Department of Electrical Engineering}, % 請確認您的科系為Department of xxx或Institute of xxx
  title         = {論文中文標題}, %論文中文標題
  title*        = {Thesis English Title}, %論文英文標題
  author        = {姓名},  % 請填寫中文姓名
  author*       = {Name},  % 請填寫英文姓名
  ID            = {RXXXXXXXX},  % 學號
  advisor       = {黃俊郎},
  advisor*      = {Jiun-Lang Huang},
  % date          = {2020-05-01},         % 若註解掉,則預設為當天
  % oral-date     = {2020-05-01},         % 若註解掉,則預設為當天
  DOI           = {10.6342/NTU202XXXXXXX},
  keywords      = {關鍵字1、關鍵字2、關鍵字3}, % 中文關鍵字,請用頓號「、」分隔
  keywords*     = {keyword 1, keyword 2, keyword 3},  % 英文關鍵字,請用逗號「,」分隔
}

% --------------------------------------------------
% 加載套件(Include Packages)
% --------------------------------------------------

\usepackage[sort&compress]{natbib}      % 參考文獻
\usepackage{amsmath, amsthm, amssymb}   % 數學環境
\usepackage{ulem, CJKulem}              % 下劃線、雙下劃線與波浪紋效果
\usepackage{booktabs}                   % 改善表格設置
\usepackage{multirow}                   % 合併儲存格
\usepackage{diagbox}                    % 插入表格反斜線
\usepackage{array}                      % 調整表格高度
\usepackage{longtable}                  % 支援跨頁長表格
\usepackage{paralist}                   % 列表環境
\usepackage{tabularx}                  % 自動調整欄寬的表格
\usepackage{makecell}

\usepackage{lipsum}                     % 英文亂字
\usepackage{zhlipsum}                   % 中文亂字

% --------------------------------------------------
% 套件設定(Packages Settings)
% --------------------------------------------------
